% Options for packages loaded elsewhere
\PassOptionsToPackage{unicode}{hyperref}
\PassOptionsToPackage{hyphens}{url}
\PassOptionsToPackage{dvipsnames,svgnames*,x11names*}{xcolor}
%
\documentclass[
]{book}
\usepackage{lmodern}
\usepackage{amssymb,amsmath}
\usepackage{ifxetex,ifluatex}
\ifnum 0\ifxetex 1\fi\ifluatex 1\fi=0 % if pdftex
  \usepackage[T1]{fontenc}
  \usepackage[utf8]{inputenc}
  \usepackage{textcomp} % provide euro and other symbols
\else % if luatex or xetex
  \usepackage{unicode-math}
  \defaultfontfeatures{Scale=MatchLowercase}
  \defaultfontfeatures[\rmfamily]{Ligatures=TeX,Scale=1}
\fi
% Use upquote if available, for straight quotes in verbatim environments
\IfFileExists{upquote.sty}{\usepackage{upquote}}{}
\IfFileExists{microtype.sty}{% use microtype if available
  \usepackage[]{microtype}
  \UseMicrotypeSet[protrusion]{basicmath} % disable protrusion for tt fonts
}{}
\makeatletter
\@ifundefined{KOMAClassName}{% if non-KOMA class
  \IfFileExists{parskip.sty}{%
    \usepackage{parskip}
  }{% else
    \setlength{\parindent}{0pt}
    \setlength{\parskip}{6pt plus 2pt minus 1pt}}
}{% if KOMA class
  \KOMAoptions{parskip=half}}
\makeatother
\usepackage{xcolor}
\IfFileExists{xurl.sty}{\usepackage{xurl}}{} % add URL line breaks if available
\IfFileExists{bookmark.sty}{\usepackage{bookmark}}{\usepackage{hyperref}}
\hypersetup{
  pdftitle={Numerik des Maschinellen Lernens},
  pdfauthor={Jan Heiland},
  colorlinks=true,
  linkcolor=Maroon,
  filecolor=Maroon,
  citecolor=Blue,
  urlcolor=purple,
  pdfcreator={LaTeX via pandoc}}
\urlstyle{same} % disable monospaced font for URLs
\usepackage{longtable,booktabs}
% Correct order of tables after \paragraph or \subparagraph
\usepackage{etoolbox}
\makeatletter
\patchcmd\longtable{\par}{\if@noskipsec\mbox{}\fi\par}{}{}
\makeatother
% Allow footnotes in longtable head/foot
\IfFileExists{footnotehyper.sty}{\usepackage{footnotehyper}}{\usepackage{footnote}}
\makesavenoteenv{longtable}
\usepackage{graphicx}
\makeatletter
\def\maxwidth{\ifdim\Gin@nat@width>\linewidth\linewidth\else\Gin@nat@width\fi}
\def\maxheight{\ifdim\Gin@nat@height>\textheight\textheight\else\Gin@nat@height\fi}
\makeatother
% Scale images if necessary, so that they will not overflow the page
% margins by default, and it is still possible to overwrite the defaults
% using explicit options in \includegraphics[width, height, ...]{}
\setkeys{Gin}{width=\maxwidth,height=\maxheight,keepaspectratio}
% Set default figure placement to htbp
\makeatletter
\def\fps@figure{htbp}
\makeatother
\setlength{\emergencystretch}{3em} % prevent overfull lines
\providecommand{\tightlist}{%
  \setlength{\itemsep}{0pt}\setlength{\parskip}{0pt}}
\setcounter{secnumdepth}{5}
\usepackage{xcolor}
\definecolor{jhsc}{HTML}{1f57c7}
\newenvironment {JHSAYS} [0] {\begin{quote}\color{jhsc}} {\end{quote}}

\title{Numerik des Maschinellen Lernens}
\author{Jan Heiland}
\date{TU Ilmenau -- Sommersemester 2024}

\usepackage{amsthm}
\newtheorem{theorem}{Theorem}[chapter]
\newtheorem{lemma}{Lemma}[chapter]
\newtheorem{corollary}{Corollary}[chapter]
\newtheorem{proposition}{Proposition}[chapter]
\newtheorem{conjecture}{Conjecture}[chapter]
\theoremstyle{definition}
\newtheorem{definition}{Definition}[chapter]
\theoremstyle{definition}
\newtheorem{example}{Example}[chapter]
\theoremstyle{definition}
\newtheorem{exercise}{Exercise}[chapter]
\theoremstyle{definition}
\newtheorem{hypothesis}{Hypothesis}[chapter]
\theoremstyle{remark}
\newtheorem*{remark}{Remark}
\newtheorem*{solution}{Solution}
\begin{document}
\maketitle

{
\hypersetup{linkcolor=}
\setcounter{tocdepth}{1}
\tableofcontents
}
\hypertarget{vorwort}{%
\chapter*{Vorwort}\label{vorwort}}
\addcontentsline{toc}{chapter}{Vorwort}

Das ist ein Aufschrieb der parallel zur Vorlesung erweitert wird.

Korrekturen und Wünsche immer gerne als \emph{issues} oder \emph{pull requests} ans \href{https://github.com/highlando/script-nmdl}{github-repo}.

\hypertarget{introduction}{%
\chapter{Introduction}\label{introduction}}

What is \emph{Numerical Methods for Machine Learning}? (ML)

In short, for the training of an ML model, a computer steps through millions of instructions that are formulated in terms of mathematical expressions.
Same holds for the evaluation of such a model.

In order to describe what is happening and for the analysis later, we introduce the general concepts

\begin{itemize}
\tightlist
\item
  algorithm
\item
  methods
\item
  accuracy
\item
  stability
\item
  exception
\end{itemize}

some of which are classical \emph{numerical analysis}.

Curiously, the term \emph{algorithm} is similarly intuitive and abstract. It took great efforts to come up with a general and concise definition that would meet requirements and limitations of all fields (ranging from, say, \emph{cooking recipes} to the analysis of of \emph{formal languages}).

\begin{definition}[Algorithm]
\protect\hypertarget{def:algorithm}{}\label{def:algorithm}A problem solution procedure is called an \emph{algorithm} if, and only if, there exists a \emph{Turing machine} that is equivalent to the procedure and that, for every input for which a solution exists, \emph{stops}.
\end{definition}

This definition is not too helpful in its generality -- we haven't even defined what is a Turing machine.

\leavevmode\hypertarget{rem-coors}{}%
\begin{JHSAYS}
A \emph{Turing machine} can be described as a machine that reads a strip of instructions and that can write onto this strip. Depending on what it reads it may move forward, move backward, or stop (when the strip has reached a predefined state). The beauty is that this setup can be put into an entirely mathematical framework.

\end{JHSAYS}

It is more helpful and more common, to look at the implications of this definition to check if a procedure has at least the necessary conditions for being an algorithm

\begin{itemize}
\tightlist
\item
  The algorithm is described by finitely many instructions (finiteness).
\item
  Every step is \emph{feasible}.
\item
  The algorithm requires a finite amount of memory.
\item
  It will finish after finitely many steps.
\item
  At every step, the next step is uniquely defined (\emph{deterministic}).
\item
  For the same initial state, it will stop at the same final state (\emph{determined}).
\end{itemize}

Thus, an informal good-practice definition of an algorithm could be

\begin{definition}[Algorithm -- informally]
\protect\hypertarget{def:info-algorithm}{}\label{def:info-algorithm}An procedure of finitely many instructions that deterministically defines a determined solution to a problem -- if it exsists -- in finitely many steps is called an algorithm.
\end{definition}

\leavevmode\hypertarget{rem-coors}{}%
\begin{JHSAYS}
Note how some properties (like finitely many instructions) are assumed a-priori.

\end{JHSAYS}

Further reading:

\begin{itemize}
\tightlist
\item
  \href{https://de.wikipedia.org/wiki/Algorithmus\#Definition}{wikipedia:Algorithmus}
\end{itemize}

\hypertarget{referenzen}{%
\chapter*{Referenzen}\label{referenzen}}
\addcontentsline{toc}{chapter}{Referenzen}

\end{document}

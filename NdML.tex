% Options for packages loaded elsewhere
\PassOptionsToPackage{unicode}{hyperref}
\PassOptionsToPackage{hyphens}{url}
\PassOptionsToPackage{dvipsnames,svgnames*,x11names*}{xcolor}
%
\documentclass[
]{book}
\usepackage{lmodern}
\usepackage{amssymb,amsmath}
\usepackage{ifxetex,ifluatex}
\ifnum 0\ifxetex 1\fi\ifluatex 1\fi=0 % if pdftex
  \usepackage[T1]{fontenc}
  \usepackage[utf8]{inputenc}
  \usepackage{textcomp} % provide euro and other symbols
\else % if luatex or xetex
  \usepackage{unicode-math}
  \defaultfontfeatures{Scale=MatchLowercase}
  \defaultfontfeatures[\rmfamily]{Ligatures=TeX,Scale=1}
\fi
% Use upquote if available, for straight quotes in verbatim environments
\IfFileExists{upquote.sty}{\usepackage{upquote}}{}
\IfFileExists{microtype.sty}{% use microtype if available
  \usepackage[]{microtype}
  \UseMicrotypeSet[protrusion]{basicmath} % disable protrusion for tt fonts
}{}
\makeatletter
\@ifundefined{KOMAClassName}{% if non-KOMA class
  \IfFileExists{parskip.sty}{%
    \usepackage{parskip}
  }{% else
    \setlength{\parindent}{0pt}
    \setlength{\parskip}{6pt plus 2pt minus 1pt}}
}{% if KOMA class
  \KOMAoptions{parskip=half}}
\makeatother
\usepackage{xcolor}
\IfFileExists{xurl.sty}{\usepackage{xurl}}{} % add URL line breaks if available
\IfFileExists{bookmark.sty}{\usepackage{bookmark}}{\usepackage{hyperref}}
\hypersetup{
  pdftitle={Numerik des Maschinellen Lernens},
  pdfauthor={Jan Heiland},
  colorlinks=true,
  linkcolor=Maroon,
  filecolor=Maroon,
  citecolor=Blue,
  urlcolor=purple,
  pdfcreator={LaTeX via pandoc}}
\urlstyle{same} % disable monospaced font for URLs
\usepackage{color}
\usepackage{fancyvrb}
\newcommand{\VerbBar}{|}
\newcommand{\VERB}{\Verb[commandchars=\\\{\}]}
\DefineVerbatimEnvironment{Highlighting}{Verbatim}{commandchars=\\\{\}}
% Add ',fontsize=\small' for more characters per line
\usepackage{framed}
\definecolor{shadecolor}{RGB}{248,248,248}
\newenvironment{Shaded}{\begin{snugshade}}{\end{snugshade}}
\newcommand{\AlertTok}[1]{\textcolor[rgb]{0.94,0.16,0.16}{#1}}
\newcommand{\AnnotationTok}[1]{\textcolor[rgb]{0.56,0.35,0.01}{\textbf{\textit{#1}}}}
\newcommand{\AttributeTok}[1]{\textcolor[rgb]{0.77,0.63,0.00}{#1}}
\newcommand{\BaseNTok}[1]{\textcolor[rgb]{0.00,0.00,0.81}{#1}}
\newcommand{\BuiltInTok}[1]{#1}
\newcommand{\CharTok}[1]{\textcolor[rgb]{0.31,0.60,0.02}{#1}}
\newcommand{\CommentTok}[1]{\textcolor[rgb]{0.56,0.35,0.01}{\textit{#1}}}
\newcommand{\CommentVarTok}[1]{\textcolor[rgb]{0.56,0.35,0.01}{\textbf{\textit{#1}}}}
\newcommand{\ConstantTok}[1]{\textcolor[rgb]{0.00,0.00,0.00}{#1}}
\newcommand{\ControlFlowTok}[1]{\textcolor[rgb]{0.13,0.29,0.53}{\textbf{#1}}}
\newcommand{\DataTypeTok}[1]{\textcolor[rgb]{0.13,0.29,0.53}{#1}}
\newcommand{\DecValTok}[1]{\textcolor[rgb]{0.00,0.00,0.81}{#1}}
\newcommand{\DocumentationTok}[1]{\textcolor[rgb]{0.56,0.35,0.01}{\textbf{\textit{#1}}}}
\newcommand{\ErrorTok}[1]{\textcolor[rgb]{0.64,0.00,0.00}{\textbf{#1}}}
\newcommand{\ExtensionTok}[1]{#1}
\newcommand{\FloatTok}[1]{\textcolor[rgb]{0.00,0.00,0.81}{#1}}
\newcommand{\FunctionTok}[1]{\textcolor[rgb]{0.00,0.00,0.00}{#1}}
\newcommand{\ImportTok}[1]{#1}
\newcommand{\InformationTok}[1]{\textcolor[rgb]{0.56,0.35,0.01}{\textbf{\textit{#1}}}}
\newcommand{\KeywordTok}[1]{\textcolor[rgb]{0.13,0.29,0.53}{\textbf{#1}}}
\newcommand{\NormalTok}[1]{#1}
\newcommand{\OperatorTok}[1]{\textcolor[rgb]{0.81,0.36,0.00}{\textbf{#1}}}
\newcommand{\OtherTok}[1]{\textcolor[rgb]{0.56,0.35,0.01}{#1}}
\newcommand{\PreprocessorTok}[1]{\textcolor[rgb]{0.56,0.35,0.01}{\textit{#1}}}
\newcommand{\RegionMarkerTok}[1]{#1}
\newcommand{\SpecialCharTok}[1]{\textcolor[rgb]{0.00,0.00,0.00}{#1}}
\newcommand{\SpecialStringTok}[1]{\textcolor[rgb]{0.31,0.60,0.02}{#1}}
\newcommand{\StringTok}[1]{\textcolor[rgb]{0.31,0.60,0.02}{#1}}
\newcommand{\VariableTok}[1]{\textcolor[rgb]{0.00,0.00,0.00}{#1}}
\newcommand{\VerbatimStringTok}[1]{\textcolor[rgb]{0.31,0.60,0.02}{#1}}
\newcommand{\WarningTok}[1]{\textcolor[rgb]{0.56,0.35,0.01}{\textbf{\textit{#1}}}}
\usepackage{longtable,booktabs}
% Correct order of tables after \paragraph or \subparagraph
\usepackage{etoolbox}
\makeatletter
\patchcmd\longtable{\par}{\if@noskipsec\mbox{}\fi\par}{}{}
\makeatother
% Allow footnotes in longtable head/foot
\IfFileExists{footnotehyper.sty}{\usepackage{footnotehyper}}{\usepackage{footnote}}
\makesavenoteenv{longtable}
\usepackage{graphicx}
\makeatletter
\def\maxwidth{\ifdim\Gin@nat@width>\linewidth\linewidth\else\Gin@nat@width\fi}
\def\maxheight{\ifdim\Gin@nat@height>\textheight\textheight\else\Gin@nat@height\fi}
\makeatother
% Scale images if necessary, so that they will not overflow the page
% margins by default, and it is still possible to overwrite the defaults
% using explicit options in \includegraphics[width, height, ...]{}
\setkeys{Gin}{width=\maxwidth,height=\maxheight,keepaspectratio}
% Set default figure placement to htbp
\makeatletter
\def\fps@figure{htbp}
\makeatother
\setlength{\emergencystretch}{3em} % prevent overfull lines
\providecommand{\tightlist}{%
  \setlength{\itemsep}{0pt}\setlength{\parskip}{0pt}}
\setcounter{secnumdepth}{5}
\usepackage{xcolor}
\definecolor{jhsc}{HTML}{1f57c7}
\newenvironment {JHSAYS} [0] {\begin{quote}\color{jhsc}} {\end{quote}}
\newlength{\cslhangindent}
\setlength{\cslhangindent}{1.5em}
\newenvironment{cslreferences}%
  {}%
  {\par}

\title{Numerik des Maschinellen Lernens}
\author{Jan Heiland}
\date{TU Ilmenau -- Sommersemester 2024}

\usepackage{amsthm}
\newtheorem{theorem}{Theorem}[chapter]
\newtheorem{lemma}{Lemma}[chapter]
\newtheorem{corollary}{Corollary}[chapter]
\newtheorem{proposition}{Proposition}[chapter]
\newtheorem{conjecture}{Conjecture}[chapter]
\theoremstyle{definition}
\newtheorem{definition}{Definition}[chapter]
\theoremstyle{definition}
\newtheorem{example}{Example}[chapter]
\theoremstyle{definition}
\newtheorem{exercise}{Exercise}[chapter]
\theoremstyle{definition}
\newtheorem{hypothesis}{Hypothesis}[chapter]
\theoremstyle{remark}
\newtheorem*{remark}{Remark}
\newtheorem*{solution}{Solution}
\begin{document}
\maketitle

{
\hypersetup{linkcolor=}
\setcounter{tocdepth}{1}
\tableofcontents
}
\hypertarget{vorwort}{%
\chapter*{Vorwort}\label{vorwort}}
\addcontentsline{toc}{chapter}{Vorwort}

Das ist ein Aufschrieb der parallel zur Vorlesung erweitert wird.

Korrekturen und Wünsche immer gerne als \emph{issues} oder \emph{pull requests} ans \href{https://github.com/highlando/script-nmdl}{github-repo}.

\hypertarget{einfuxfchrung}{%
\chapter{Einführung}\label{einfuxfchrung}}

Was sind \emph{Numerische Methoden für Maschinelles Lernen} (ML)?

Kurz gesagt, beim Training eines ML-Modells durchläuft ein Computer Millionen von Anweisungen, die in Form mathematischer Ausdrücke formuliert sind. Gleiches gilt für die Bewertung eines solchen Modells.
Dann stellen sich Fragen wie \emph{wird es einen Punkt geben, an dem das Training endet?} und \emph{wird das Modell genau sein?}.

Um zu beschreiben, was passiert, und für die spätere Analyse führen wir die allgemeinen Konzepte von

\begin{itemize}
\tightlist
\item
  Algorithmus
\item
  Konsistenz/Genauigkeit
\item
  Stabilität
\item
  Rechenaufwand
\end{itemize}

ein, von denen einige klassische \emph{numerische Analysis} sind.

\hypertarget{was-ist-ein-algorithmus}{%
\section{Was ist ein Algorithmus}\label{was-ist-ein-algorithmus}}

Interessanterweise ist der Begriff \emph{Algorithmus} zugleich intuitiv und abstrakt. Es bedurfte großer Anstrengungen, um eine allgemeine und wohlgestellte Definition zu finden, die den Anforderungen und Einschränkungen aller Bereiche gerecht wird (von \emph{Kochrezepten} bis zur Analyse von \emph{formalen Sprachen}).

\begin{definition}[Algorithmus]
\protect\hypertarget{def:algorithm}{}\label{def:algorithm}Ein Problemlösungsverfahren wird als \emph{Algorithmus} bezeichnet, genau dann wenn es eine \emph{Turing-Maschine} gibt, die dem Verfahren entspricht und die, für jede Eingabe, für die eine Lösung existiert, \emph{anhalten} wird.
\end{definition}

Diese Definition ist in ihrer Allgemeinheit nicht allzu hilfreich - wir haben noch nicht einmal definiert, was eine Turing-Maschine ist.

\leavevmode\hypertarget{rem-coors}{}%
\begin{JHSAYS}
Eine \emph{Turing-Maschine} kann als eine Maschine beschrieben werden, die ein Band von Anweisungen liest und auf dieses Band schreiben kann. Abhängig davon, was sie liest, kann sie vorwärts bewegen, rückwärts bewegen oder anhalten (wenn das Band einen vordefinierten Zustand erreicht hat). Das Schöne daran ist, dass dieses Setup in einen vollständig mathematischen Rahmen gestellt werden kann.

\end{JHSAYS}

Hilfreicher und gebräuchlicher ist es, die Implikationen dieser Definition zu betrachten, um zu überprüfen, ob ein Verfahren zumindest die notwendigen Bedingungen für einen Algorithmus erfüllt

\begin{itemize}
\tightlist
\item
  Der Algorithmus wird durch endlich viele Anweisungen beschrieben (Endlichkeit).
\item
  Jeder Schritt ist \emph{durchführbar}.
\item
  Der Algorithmus erfordert eine endliche Menge an Speicher.
\item
  Er wird nach endlich vielen Schritten beendet.
\item
  In jedem Schritt ist der nächste Schritt eindeutig definiert (\emph{Determiniertheit}).
\item
  Für denselben Anfangszustand wird er im selben Endzustand anhalten (\emph{Bestimmtheit}).
\end{itemize}

Somit könnte eine informelle, gute Praxisdefinition eines Algorithmus sein

\begin{definition}[Algorithmus -- informell]
\protect\hypertarget{def:info-algorithm}{}\label{def:info-algorithm}Ein Verfahren aus endlich vielen Anweisungen wird als \emph{Algorithmus} bezeichnet, wenn es eine bestimmte Lösung -- falls sie existiert -- zu einem Problem in endlich vielen Schritten berechnet.
\end{definition}

\leavevmode\hypertarget{rem-coors}{}%
\begin{JHSAYS}
Beachten Sie, wie einige Eigenschaften (wie endlich viele Anweisungen) a priori angenommen werden.

\end{JHSAYS}

Als informellere Verweise auf Algorithmen werden wir die Begriffe \textbf{\emph{(numerische) Methode}} oder \textbf{\emph{Schema}} verwenden, um ein Verfahren durch Auflistung seiner zugrundeliegenden Ideen und Unterprozeduren anzusprechen, wobei \emph{Algorithmus} sich auf eine spezifische Realisierung einer \emph{Methode} bezieht.

Weiterhin unterscheiden wir

\begin{itemize}
\tightlist
\item
  \emph{direkte} Methoden -- die die Lösung exakt berechnen (wie die Lösung eines linearen Systems durch \emph{Gauß-Elimination}) und
\item
  \emph{iterative} Methoden -- die iterativ eine Folge von Annäherungen an die Lösung berechnen (wie die Berechnung von Wurzeln mit einem \emph{Newton-Schema}).
\end{itemize}

\hypertarget{konsistenz-stabilituxe4t-genauigkeit}{%
\section{Konsistenz, Stabilität, Genauigkeit}\label{konsistenz-stabilituxe4t-genauigkeit}}

Für die Analyse numerischer Methoden werden allgemein die folgenden Begriffe verwendet:

\begin{definition}[Konsistenz]
\protect\hypertarget{def:consistency}{}\label{def:consistency}Wenn ein Algorithmus in exakter Arithmetik die Lösung des Problems mit einer gegebenen Genauigkeit berechnet, wird er als \emph{konsistent} bezeichnet.
\end{definition}

\begin{definition}[Stabilität (informell)]
\protect\hypertarget{def:stability}{}\label{def:stability}Wenn die Ausgabe eines Algorithmus kontinuierlich von Unterschieden in der Eingabe und kontinuierlich von Unterschieden in den Anweisungen abhängt, dann wird der Algorithmus als \emph{stabil} bezeichnet.
\end{definition}

Die \emph{Unterschiede in den Anweisungen} sind typischerweise auf Rundungsfehler zurückzuführen, wie sie in \emph{ungenauer Arithmetik} (oft auch als \emph{Gleitkommaarithmetik} bezeichnet) auftreten.

\leavevmode\hypertarget{rem-coors}{}%
\begin{JHSAYS}
Man könnte sagen, dass ein Algorithmus konsistent ist, wenn \emph{er das Richtige tut} und dass er stabil ist, \emph{wenn er trotz beliebiger kleiner Ungenauigkeiten funktioniert}. Wenn ein Algorithmus konsistent und stabil ist, wird er oft als \emph{konvergent} bezeichnet, um auszudrücken, dass er schließlich die Lösung auch in ungenauer Arithmetik berechnen wird.

\end{JHSAYS}

Beachten Sie, dass Begriffe wie

\begin{itemize}
\tightlist
\item
  \emph{Genauigkeit} -- wie nahe die berechnete Ausgabe der tatsächlichen Lösung kommt oder
\item
  \emph{Konvergenz} -- wie schnell (typischerweise in Bezug auf den Rechenaufwand) der Algorithmus sich der tatsächlichen Lösung nähert
\end{itemize}

keine intrinsischen Eigenschaften eines Algorithmus sind, da sie von dem zu lösenden Problem abhängen.
Man kann jedoch von \emph{Konsistenzordnung} eines Algorithmus sprechen, um die erwartete Genauigkeit für eine Klasse von Problemen zu spezifizieren, und einen Algorithmus als konvergent einer bestimmten Ordnung bezeichnen, wenn er zusätzlich stabil ist.

\hypertarget{rechenkomplexituxe4t}{%
\section{Rechenkomplexität}\label{rechenkomplexituxe4t}}

Die \emph{Rechenkomplexität} eines Algorithmus ist sowohl theoretisch (um abzuschätzen, wie der Aufwand mit beispielsweise der Größe des Problems skaliert) als auch praktisch (um zu sagen, wie lange das Verfahren dauern wird und welche Kosten in Bezug auf CPU-Zeit oder Speichernutzung es generieren wird) wichtig.

Typischerweise wird die Komplexität durch Zählen der elementaren Operationen gemessen -- wir werden stets die Ausführung einer Grundrechenart als eine Operation zählen.

\leavevmode\hypertarget{rem-flops}{}%
\begin{JHSAYS}
Die Definition einer \emph{elementaren Operation} auf einem Computer ist nicht universal, da viele Faktoren hier reinspielen. Gerne werden \emph{FLOP}
s angeführt, was für \emph{floating point operations} steht. Allerdings ist es wiederum sehr verschieden auf verschiedenen Prozessoren wieviele FLOPs für eine Multiplikation oder Addition gebraucht werden.

\end{JHSAYS}

Um die Algorithmen in Bezug auf Komplexität versus Problemgröße zu klassifizieren, sind die folgenden Funktionsklassen hilfreich

\begin{definition}[Landau-Symbole oder große O-Notation]
\protect\hypertarget{def:landau-symbs}{}\label{def:landau-symbs}Sei \(g\colon \mathbb R^{} \to \mathbb R^{}\) und \(a\in\mathbb R^{} \cup \{-\infty, +\infty\}\). Dann sagen wir für eine Funktion \(f\colon \mathbb R \to \mathbb R^{}\), dass \(f\in O(g)\), wenn
\begin{equation*}
\limsup_{x\to a} \frac{|f(x)|}{|g(x)|} < \infty
\end{equation*}
und dass \(f\in o(g)\), wenn
\begin{equation*}
\limsup_{x\to a} \frac{|f(x)|}{|g(x)|} = 0.
\end{equation*}
\end{definition}

Der Sinn und die Funktionalität dieser Konzepte wird vielleicht deutlich, wenn man sich die typischen Anwendungen ansieht:

\begin{itemize}
\tightlist
\item
  Wenn \(h> 0\) ein Diskretisierungsparameter ist und, sagen wir, \(e(h)\) der Diskretisierungsfehler ist, dann könnten wir sagen, dass \(e(h) = O(h^2)\), wenn \emph{asymptotisch}, d.h. für immer kleinere \(h\), der Fehler mindestens so schnell wie \(h^2\) gegen \(0\) geht.
\item
  Wenn \(C(n)\) die Komplexität eines Algorithmus für eine Problemgröße \(n\) ist, dann könnten wir sagen, dass \(C(n) = O(n)\), um auszudrücken, dass die Komplexität \emph{asymptotisch}, d.h. für immer größere \(n\), mit derselben Geschwindigkeit wie die Problemgröße wächst.
\end{itemize}

Leider ist die übliche Verwendung der Landau-Symbole etwas unpräzise.

\begin{enumerate}
\def\labelenumi{\arabic{enumi}.}
\tightlist
\item
  Das oft verwendete ``\(=\)''-Zeichen ist informell und keineswegs eine Gleichheit.
\item
  Was der Grenzwert \(a\) ist, wird selten explizit erwähnt, aber glücklicherweise ist es in der Regel aus dem Kontext klar.
\end{enumerate}

Als Beispiel betrachten wir zwei verschiedene Wege, ein Polynom \(p\) vom Grad \(n\) an der Abszisse \(x\) auszuwerten, basierend auf den zwei äquivalenten Darstellungen
\begin{equation*}
\begin{split}
p(x) &= a_0 + a_1x +  a_2x^2+ \dotsm + a_nx^n \\
     &= a_0 + x(a_1 + x(a_2 + \dotsm +x(a_{n-1} + a_nx) \dotsm ))
\end{split}
\end{equation*}

Für eine direkte Implementierung der ersten Darstellung erhalten wir die Algorithmen

\begin{Shaded}
\begin{Highlighting}[]
\CommentTok{\textquotesingle{}\textquotesingle{}\textquotesingle{}Berechnung von p(x) in Standarddarstellung}
\CommentTok{\textquotesingle{}\textquotesingle{}\textquotesingle{}}
\NormalTok{n }\OperatorTok{=} \DecValTok{10}                                      \CommentTok{\# Beispielwert für n}
\NormalTok{ais }\OperatorTok{=}\NormalTok{ [(}\OperatorTok{{-}}\DecValTok{1}\NormalTok{)}\OperatorTok{**}\NormalTok{k}\OperatorTok{*}\DecValTok{1}\OperatorTok{/}\NormalTok{k }\ControlFlowTok{for}\NormalTok{ k }\KeywordTok{in} \BuiltInTok{range}\NormalTok{(}\DecValTok{1}\NormalTok{, n}\OperatorTok{+}\DecValTok{2}\NormalTok{)]  }\CommentTok{\# Liste der Beispielkoeffizienten}
\NormalTok{x }\OperatorTok{=} \DecValTok{5}                                       \CommentTok{\# Ein Beispielwert für x}
\NormalTok{cpx }\OperatorTok{=}\NormalTok{ ais[}\DecValTok{0}\NormalTok{]                                }\CommentTok{\# der Fall k=0}
\ControlFlowTok{for}\NormalTok{ k }\KeywordTok{in} \BuiltInTok{range}\NormalTok{(n):}
\NormalTok{    cpx }\OperatorTok{=}\NormalTok{ cpx }\OperatorTok{+}\NormalTok{ ais[k}\OperatorTok{+}\DecValTok{1}\NormalTok{] }\OperatorTok{*}\NormalTok{ x}\OperatorTok{**}\NormalTok{(k}\OperatorTok{+}\DecValTok{1}\NormalTok{)         }\CommentTok{\# der Beitrag des k{-}ten Schritts}
\BuiltInTok{print}\NormalTok{(}\SpecialStringTok{f\textquotesingle{}x=}\SpecialCharTok{\{x\}}\SpecialStringTok{: p(x)=}\SpecialCharTok{\{}\NormalTok{cpx}\SpecialCharTok{:.4f\}}\SpecialStringTok{\textquotesingle{}}\NormalTok{)             }\CommentTok{\# Ausgabe des Ergebnisses}
\end{Highlighting}
\end{Shaded}

Im \(k\)-ten Schritt benötigt der Algorithmus eine Addition (wenn wir auch die Initialisierung als Addition zählen) und \(k\) Multiplikationen. Das ergibt eine Gesamtkomplexität von
\begin{equation*}
C(n) = \sum_{k=0}^n(1+k) = n+1 + \frac{n(n-1)}{2} = 1 + \frac n2 + \frac{n^2}2 = O(n^2)
\end{equation*}

Für die zweite Darstellung können wir das sogenannte \emph{Horner-Schema} implementieren, das lauten würde

\begin{Shaded}
\begin{Highlighting}[]
\CommentTok{\textquotesingle{}\textquotesingle{}\textquotesingle{}Berechnung von p(x) mit dem Horner{-}Schema}
\CommentTok{\textquotesingle{}\textquotesingle{}\textquotesingle{}}
\NormalTok{n }\OperatorTok{=} \DecValTok{10}                                      \CommentTok{\# Beispielwert für n}
\NormalTok{ais }\OperatorTok{=}\NormalTok{ [(}\OperatorTok{{-}}\DecValTok{1}\NormalTok{)}\OperatorTok{**}\NormalTok{k}\OperatorTok{*}\DecValTok{1}\OperatorTok{/}\NormalTok{k }\ControlFlowTok{for}\NormalTok{ k }\KeywordTok{in} \BuiltInTok{range}\NormalTok{(}\DecValTok{1}\NormalTok{, n}\OperatorTok{+}\DecValTok{2}\NormalTok{)]  }\CommentTok{\# Liste der Beispielkoeffizienten}
\NormalTok{x }\OperatorTok{=} \DecValTok{5}                                       \CommentTok{\# Ein Beispielwert für x}
\NormalTok{cpx }\OperatorTok{=}\NormalTok{ ais[n]                                }\CommentTok{\# der Fall k=n}
\ControlFlowTok{for}\NormalTok{ k }\KeywordTok{in} \BuiltInTok{reversed}\NormalTok{(}\BuiltInTok{range}\NormalTok{(n)):                }
\NormalTok{    cpx }\OperatorTok{=}\NormalTok{ ais[k] }\OperatorTok{+}\NormalTok{ x}\OperatorTok{*}\NormalTok{cpx                    }\CommentTok{\# der Beitrag des k{-}ten Schritts}
\BuiltInTok{print}\NormalTok{(}\SpecialStringTok{f\textquotesingle{}x=}\SpecialCharTok{\{x\}}\SpecialStringTok{: p(x)=}\SpecialCharTok{\{}\NormalTok{cpx}\SpecialCharTok{:.4f\}}\SpecialStringTok{\textquotesingle{}}\NormalTok{)             }\CommentTok{\# Ausgabe des Ergebnisses}
\end{Highlighting}
\end{Shaded}

Insgesamt benötigt dieses Schema \(n+1\) Additionen und \(n\) Multiplikationen, d.h. \(2n+1\) FLOPs, so dass wir sagen können, dass \emph{dieser Algorithmus \(O(n)\) ist}.

\hypertarget{literatur}{%
\section{Literatur}\label{literatur}}

\begin{itemize}
\tightlist
\item
  (Nocedal and Wright \protect\hyperlink{ref-NocW06}{2006}): Ein gut lesbares Buch zur Optimierung.
\end{itemize}

\hypertarget{uxfcbungen}{%
\section{Übungen}\label{uxfcbungen}}

\begin{enumerate}
\def\labelenumi{\arabic{enumi}.}
\item
  Vergleichen Sie die beiden Implementierungen zur Auswertung eines Polynoms, indem Sie die Komplexität als Funktion von \(n\) darstellen und die benötigte CPU-Zeit für eine Beispielauswertung im Vergleich zu \(n\) messen und darstellen.
\item
  Zeigen Sie, dass es für \(f\in O(g)\) mit \(f\geq 0\) und \(g> 0\) eine Konstante \(C\) gibt, sodass \(f(n)=h(n) + Cg(n)\) mit \(h\in o(g)\). \emph{Bemerkung: diese Relation ist die Rechtfertigung für die eigentlich inkorrekte Schreibweise \(f=O(g)\)}.
\item
  Ermitteln Sie experimentell \emph{die Ordnung} (d.h. den Exponent \(x\) in \(O(n^x)\)) und \emph{die Konstante} \(C\) (s.o.) für die Laufzeit \(t(n)\) der in \texttt{scipy.linalg.cholesky} implementierten Cholesky Zerlegung der Bandmatrix \texttt{A\_n} aus dem folgenden Code Beispiel
\end{enumerate}

\begin{Shaded}
\begin{Highlighting}[]
\ImportTok{import}\NormalTok{ numpy }\ImportTok{as}\NormalTok{ np}
\ImportTok{from}\NormalTok{ scipy.linalg }\ImportTok{import}\NormalTok{ cholesky}
\ImportTok{from}\NormalTok{ time }\ImportTok{import}\NormalTok{ time}
\NormalTok{n }\OperatorTok{=} \DecValTok{10}                                  \CommentTok{\# example problem size}
\NormalTok{A\_n }\OperatorTok{=} \OperatorTok{{-}}\DecValTok{1}\OperatorTok{*}\NormalTok{np.diag(np.ones(n}\OperatorTok{{-}}\DecValTok{1}\NormalTok{), }\OperatorTok{{-}}\DecValTok{1}\NormalTok{) }\OperatorTok{+} \OperatorTok{\textbackslash{}}  \CommentTok{\# a tridiagonal band matrix}
    \DecValTok{2}\OperatorTok{*}\NormalTok{np.diag(np.ones(n), }\DecValTok{0}\NormalTok{) }\OperatorTok{+} \OperatorTok{\textbackslash{}}
    \OperatorTok{{-}}\DecValTok{1}\OperatorTok{*}\NormalTok{np.diag(np.ones(n}\OperatorTok{{-}}\DecValTok{1}\NormalTok{), }\DecValTok{1}\NormalTok{)}
\NormalTok{tic }\OperatorTok{=}\NormalTok{ time()                            }\CommentTok{\# start the timer}
\NormalTok{\_ }\OperatorTok{=}\NormalTok{ cholesky(A\_n)                       }\CommentTok{\# perform the computation}
\NormalTok{toc }\OperatorTok{=}\NormalTok{ time()                            }\CommentTok{\# stop the timer}
\BuiltInTok{print}\NormalTok{(}\SpecialStringTok{f\textquotesingle{}n: }\SpecialCharTok{\{n\}}\SpecialStringTok{ {-}{-} t\_n: }\SpecialCharTok{\{}\NormalTok{toc}\OperatorTok{{-}}\NormalTok{tic}\SpecialCharTok{:.4e\}}\SpecialStringTok{\textquotesingle{}}\NormalTok{)}
\end{Highlighting}
\end{Shaded}

\emph{Hinweis: Hier geht es um die Methodik und um eine sinnvolle Interpretation der Ergebnisse. Es kann gut sein, dass die Ergebnisse auf verschiedenen Rechnern verschieden ausfallen. Außerdem können für große \(n\) (wenn der Exponent und die Konstante am besten sichtbar sind) auf einmal bspw. ein zu voller Arbeitsspeicher die Berechnung negativ beeinflussen.}

\begin{enumerate}
\def\labelenumi{\arabic{enumi}.}
\setcounter{enumi}{3}
\tightlist
\item
  Diskutieren Sie, wie Laufzeitmessungen (bspw. zur Komplexitätsanalyse eines Verfahrens) aufgesetzt werden sollten, um reproduzierbare Ergebnisse zu erhalten. Was sollte dokumentiert werden, damit dritte Personen die Ergebnisse einordnen und ggf. reproduzieren können.
\end{enumerate}

Weiterführende Literatur:

\begin{itemize}
\tightlist
\item
  \href{https://de.wikipedia.org/wiki/Algorithmus\#Definition}{wikipedia:Algorithmus}
\end{itemize}

\hypertarget{fehler-und-konditionierung}{%
\chapter{Fehler und Konditionierung}\label{fehler-und-konditionierung}}

\def\kij{(\kappa_{A,x})_{ij}}

Berechnungen auf einem Computer verursachen unvermeidlich Fehler, und die Effizienz oder Leistung von Algorithmen ist immer das Verhältnis von Kosten zu Genauigkeit.

Zum Beispiel:

\begin{itemize}
\item
  Allein aus der Betrachtung von Rundungsfehlern kann die Genauigkeit einfach und signifikant verbessert werden, indem auf \emph{Langzahlarithmetik} zurückgegriffen wird, was jedoch höhere Speicheranforderungen und eine höhere Rechenlast mit sich bringt.
\item
  In iterativen Verfahren können Speicher und Rechenaufwand leicht eingespart werden, indem die Iteration in einem frühen Stadium gestoppt wird - natürlich auf Kosten einer weniger genauen Lösungsapproximation.
\end{itemize}

\leavevmode\hypertarget{rem-accu-iter}{}%
\begin{JHSAYS}
Beide, irgendwie trivialen Beobachtungen sind grundlegende Bestandteile des Trainings neuronaler Netzwerke. Erstens wurde beobachtet, dass Zahldarstellungen mit \emph{einfacher Genauigkeit} (im Vergleich zum gängigen \emph{double precision}) Rechenkosten sparen kann, mit nur geringen Auswirkungen auf die Genauigkeit. Zweitens ist das Training ein iterativer Prozess mit oft langsamer Konvergenz, sodass der richtige Zeitpunkt für einen vorzeitigen Abbruch des Trainings entscheidend ist.

\end{JHSAYS}

\hypertarget{fehler}{%
\section{Fehler}\label{fehler}}

\begin{definition}[Absolute und relative Fehler]
\protect\hypertarget{def:errors}{}\label{def:errors}Sei \(x\in\mathbb R^{}\) die interessierende Größe und \(\tilde x \in \mathbb R^{}\) eine Annäherung daran. Dann wird der \emph{absolute Fehler} definiert als \(|\delta x|:=|\tilde x- x|\) und der \emph{relative Fehler} als \(\frac{|\delta x|}{|x|}=\frac{|\tilde x- x|}{|x|}\).
\end{definition}

\leavevmode\hypertarget{rem-rel-abs-err}{}%
\begin{JHSAYS}
Generell wird der relative Fehler bevorzugt, da er den gemessenen Fehler in den richtigen Bezug setzt. Zum Beispiel kann ein absoluter Fehler von \(10\) km/h je nach Kontext groß oder klein sein.

\end{JHSAYS}

\hypertarget{kondition}{%
\section{Kondition}\label{kondition}}

Als Nächstes definieren wir die \emph{Kondition} eines Problems \(A\) und analog eines Algorithmus (der das Problem löst). Dafür lassen wir \(x\) einen Parameter/Eingabe des Problems sein und \(y=A(x)\) die entsprechende Lösung/Ausgabe. Die Kondition ist ein Maß dafür, inwieweit eine Änderung \(x+\delta x\) in der Eingabe die resultierende relative Änderung in der Ausgabe beeinflusst. Dafür betrachten wir
\begin{equation*}
\delta y = \tilde y - y = A(\tilde x) - A(x) = A(x+\delta x) - A(x)
\end{equation*}
welches nach Division durch \(y=A(x)\) und Erweiterung durch \(x\,\delta x\) wird zu
\begin{equation*}
\frac{\delta y}{y} = \frac{A(x+\delta x)-A(x)}{\delta x}\frac{x}{A(x)}\frac{\delta x}{x}.
\end{equation*}
Für infinitesimal kleine \(\delta x\) wird der Differenzenquotient \(\frac{A(x+\delta x)-A(x)}{\delta x}\) zur Ableitung \(\frac{\partial A}{\partial x}(x)\), so dass wir die Kondition des Problems/Algorithmus bei \(x\) abschätzen können durch

\begin{equation}
\frac{|\delta y|}{|y|} \approx |\frac{\partial A}{\partial x}(x)|\frac{|x|}{|A(x)|}\frac{|\delta x|}{|x|}=:\kappa_{A,x}\frac{|\delta x|}{|x|}.
\label{eq:eqn-scalar-cond}
\end{equation}

Wir nennen \(\kappa_{A,x}\) die Konditionszahl.

Für vektorwertige Probleme/Algorithmen können wir die Konditionszahl darüber definieren, wie eine Differenz in der \(j\)-ten Eingabekomponente \(x_j\) die \(i\)-te Komponente \(y_i=A_i(x)\) der Ausgabe beeinflusst.

\begin{definition}[Konditionszahl]
\protect\hypertarget{def:condition}{}\label{def:condition}Für ein Problem/Algorithmus \(A\colon \mathbb R^{n}\to \mathbb R^{m}\) nennen wir
\begin{equation*}
(\kappa_{A,x})_{ij} := \frac{\partial A_i}{\partial x_j}(x) \frac{x_j}{A_i(x)}
\end{equation*}
die partielle \emph{Konditionszahl} des Problems. Ein Problem wird als \emph{gut konditioniert} bezeichnet, wenn \(|(\kappa_{A,x})_{ij}|\approx 1\) und als \emph{schlecht konditioniert}, wenn \(|(\kappa_{A,x})_{ij}\gg 1\), für alle \(i=1,\dotsc,m\) und \(j=1,\dotsc,m\).
\end{definition}

\leavevmode\hypertarget{rem-vector-valued-cond}{}%
\begin{JHSAYS}
Anstatt die skalaren Komponentenfunktionen von \(A\colon \mathbb R^{n} \to \mathbb R^{m}\) zu verwenden, kann man die Berechnungen, die zu \eqref{eq:eqn-scalar-cond} geführt haben, mit vektorwertigen Größen in den entsprechenden Normen wiederholen.

\end{JHSAYS}

\hypertarget{kondition-der-grundrechenarten}{%
\section{Kondition der Grundrechenarten}\label{kondition-der-grundrechenarten}}

Da einfach jede Operation von Zahlen auf dem Computer auf die Grundrechenarten zurueckgeht, ist es wichtig sich zu vergegenwärtigen wie sich diese Basisoperationen in Bezug auf kleine Fehler verhalten.

\hypertarget{addition}{%
\subsection{Addition}\label{addition}}

\begin{Shaded}
\begin{Highlighting}[]
\KeywordTok{def}\NormalTok{ A(x, y):}
    \ControlFlowTok{return}\NormalTok{ x}\OperatorTok{+}\NormalTok{y}

\NormalTok{x, tx, y }\OperatorTok{=} \FloatTok{1.02}\NormalTok{, }\FloatTok{1.021}\NormalTok{, }\OperatorTok{{-}}\FloatTok{1.00}
\NormalTok{z }\OperatorTok{=}\NormalTok{ A(x, y)}
\NormalTok{tz }\OperatorTok{=}\NormalTok{ A(tx, y)}
\NormalTok{relerrx }\OperatorTok{=}\NormalTok{ (tx }\OperatorTok{{-}}\NormalTok{ x)}\OperatorTok{/}\NormalTok{x        }\CommentTok{\# here: 0.00098039}
\NormalTok{relerrz }\OperatorTok{=}\NormalTok{ (tz }\OperatorTok{{-}}\NormalTok{ z)}\OperatorTok{/}\NormalTok{z        }\CommentTok{\# here: 0.04999999}
\NormalTok{kondAx }\OperatorTok{=}\NormalTok{ relerrz}\OperatorTok{/}\NormalTok{relerrx    }\CommentTok{\# here: 50.9999999}
\end{Highlighting}
\end{Shaded}

In diesem Code Beispiel liegt der relative Fehler in \(x\) bei etwa \(0.01\)\% und im Ausgang \(z\) bei etwa \(5\)\%, was einer etwa \(50\)-fachen Verstärkung entspricht.
Für die Konditionszahl der Addtion \(A_y\colon x \to y+x\) gilt:
\begin{equation*}
\kappa_{A_y;x} = \frac{|x|}{|x+y|} = \frac{1}{|1+\frac{y}{x}|}.
\end{equation*}

Diese Konditionszahl kann offenbar beliebig groß werden, wenn \(x\) nah an \(-y\) liegt. Jan spricht von Auslöschung und tatsächlich lässt sich nachvollziehen, dass in diesem Fall die vorderen (korrekten) Stellen einer Zahl von einander abgezogen werden und die hinteren (möglicherweise inkorrekten) Stellen übrig bleiben.

\leavevmode\hypertarget{rem-accu-iter}{}%
\begin{JHSAYS}
Praktisch gesagt: Hantiert Jan mit Addition großer Zahlen um ein kleines Ergebnis erzielen ist das numerisch sehr ungünstig.

\end{JHSAYS}

\hypertarget{multiplikation}{%
\subsection{Multiplikation}\label{multiplikation}}

\begin{Shaded}
\begin{Highlighting}[]
\KeywordTok{def}\NormalTok{ M(x, y):}
    \ControlFlowTok{return}\NormalTok{ x}\OperatorTok{*}\NormalTok{y}

\NormalTok{x, tx, y }\OperatorTok{=} \FloatTok{1.02}\NormalTok{, }\FloatTok{1.021}\NormalTok{, }\OperatorTok{{-}}\FloatTok{1.00}
\NormalTok{z }\OperatorTok{=}\NormalTok{ M(x, y)}
\NormalTok{tz }\OperatorTok{=}\NormalTok{ M(tx, y)}
\NormalTok{relerrx }\OperatorTok{=}\NormalTok{ (tx }\OperatorTok{{-}}\NormalTok{ x)}\OperatorTok{/}\NormalTok{x        }\CommentTok{\# here: 0.00098039}
\NormalTok{relerrz }\OperatorTok{=}\NormalTok{ (tz }\OperatorTok{{-}}\NormalTok{ z)}\OperatorTok{/}\NormalTok{z        }\CommentTok{\# here: 0.00098039}
\NormalTok{kondMx }\OperatorTok{=}\NormalTok{ relerrz}\OperatorTok{/}\NormalTok{relerrx    }\CommentTok{\# here: 1.0}
\end{Highlighting}
\end{Shaded}

Das Ergebnis \texttt{1.0} für die empirsch ermittelte Konditionszahl war kein Zufall. Es gilt im Allgemeinen für \(M_y \colon x \to yx\) dass
\begin{equation*}
\kappa_{M_y;x} = |y|\frac{|x|}{|xy|} = 1.
\end{equation*}
Die Multiplikation ist also generell gut konditioniert.

\hypertarget{wurzelziehen}{%
\subsection{Wurzelziehen}\label{wurzelziehen}}

Das Berechnen der Quadratwurzel \(W\colon x \to \sqrt x\) hat die Konditionszahl \(\frac 12\). Bei Konditionszahlen kleiner als \(1\) verringert sich der relative Fehler, Jan spricht von \emph{Fehlerdämpfung}.

\hypertarget{uxfcbungen-1}{%
\section{Übungen}\label{uxfcbungen-1}}

\begin{enumerate}
\def\labelenumi{\arabic{enumi}.}
\tightlist
\item
  Leiten Sie die \emph{Konditionszahl} wie in \eqref{eq:eqn-scalar-cond} für eine vektorwertige Funktion \(A\colon \mathbb R^{n} \to \mathbb R^{m}\) her. Wo spielt eine Matrixnorm eine Rolle?
\item
  Leiten Sie mit dem selben Verfahren die Konditionszahl einer invertierbaren Matrix \(M\) her, d.h. die Kondition des Problems \(x\to y = M^{-1}x\). Wo spielt die Matrixnorm eine Rolle?
\item
  Leiten Sie die Konditionszahlen für die Operationen \emph{Division} und \emph{Quadratwurzelziehen} her.
\item
  Veranschaulichen Sie an der Darstellung des Vektors \(P=[1, 1]\) in der Standardbasis \(\{[1, 0], \,[0, 1]\}\) und in der Basis \(\{[1, 0], \,[1, 0.1]\}\) unter Verweis auf die Kondition der Addition, warum \emph{orthogonale Basen} als \emph{gut konditioniert} gelten.
\end{enumerate}

\hypertarget{nachklapp}{%
\chapter{Nachklapp}\label{nachklapp}}

Ein paar lose Beispiele wo Numerik und maschinelles Lernen sich treffen.

\begin{itemize}
\item
  Iterative Methoden

  \begin{itemize}
  \tightlist
  \item
    Konvergenz/Konvergenzraten
  \item
    stochastische Konvergenz
  \item
    lokale Extrema
  \item
    randomisierte Methoden
  \end{itemize}
\item
  Optimierung/Ausgleichsrechnung
\item
  Approximationstheorie

  \begin{itemize}
  \tightlist
  \item
    Universal Approximation Theorem
  \end{itemize}
\item
  Stabilität und Fehleranalyse

  \begin{itemize}
  \tightlist
  \item
    mixed precision Arithmetik
  \end{itemize}
\item
  Numerische lineare Algebra

  \begin{itemize}
  \tightlist
  \item
    PCA
  \item
    Support Vector Machines
  \item
    Empfehlungssysteme
  \end{itemize}
\item
  Automatisches Differenzieren

  \begin{itemize}
  \tightlist
  \item
    \emph{backward propagation} zur Gradientenberechnung
  \end{itemize}
\end{itemize}

\hypertarget{referenzen}{%
\chapter*{Referenzen}\label{referenzen}}
\addcontentsline{toc}{chapter}{Referenzen}

\hypertarget{refs}{}
\begin{cslreferences}
\leavevmode\hypertarget{ref-NocW06}{}%
Nocedal, J., Wright, S.J.: Numerical optimization. Springer (2006)
\end{cslreferences}

\end{document}
